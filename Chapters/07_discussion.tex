\chapter{Conclusions}

%appreciate the limitations of the methods employed and the results obtained by yourself and others
%understand how the broad conclusions of your thesis support, add to or conflict with previous work

% Sum up findings of four chapters

In this thesis I have analysed multiple RNA-seq datasets to uncover new insights into TDP-43 and FUS.
I have explored the physiological roles of the two proteins in RNA regulation through loss-of-function experiments.
Furthermore, by using mutant mice as models of disease, I have uncovered mechanisms by which disease-associated mutations can impart a gain of function. 
This comparison between loss and gain of functionality will be crucial for understanding the onset and course of ALS/FTD.

In chapter 3 I developed software to discover and classify the full extent of cryptic splicing repression by TDP-43 and applied it to multiple mouse and human datasets.
I demonstrated that this repression is due to a combination of TDP-43 binding motifs and strong splice site sequences co-occurring within poorly conserved introns.
This mechanism was not seen in FUS knockdown, suggesting a divergence in function for the two proteins.
Later work has shown that cryptic exon repression is seen in a range of RNA-binding proteins, including the ALS-associated MATRIN3, SFPQ and TIA1.

In chapter 4 I analysed data from a novel FUS mutant mouse which modelled aggressively early onset FUS ALS. 
The mice are heterozygous for a mutation which removes the FUS nuclear localisation signal, leading to cytoplasmic mislocalisation of the mutant FUS protein without depleting wildtype FUS from the nucleus.
By looking across two tissues and time-points I identified a specific transcriptomic signature seen only in late-adult mouse spinal cord.
This was dominated by changes in mitochondrial and ribosomal genes.
Splicing changes at this tissue and time point were restricted to FUS itself and a small number of other genes including the myelin protein \textit{Mbp} and the FUS-interacting RNA-binding protein \textit{Ewsr1}.

In chapter 5 I compared two TDP-43 mutant mice lines. 
One exhibited a loss of splicing function due to the mutation affecting an RNA-recognition motif. 
As expected, this led to the inclusion of cryptic exons.
The other had a mutation within the TDP-43 low-complexity domain, a hotspot for ALS mutations.
I demonstrated that this mutation imparts a gain of splicing function characterised by a set of aberrant constitutive exon skipping events.
Some of these ``skiptic'' exons could be identified in TDP-43 mutant human patients.
The gain of splicing function could arise from a change in TDP-43 autoregulation or from a shift in how TDP-43 interacts with other splicing factors. 

In chapter 6 I compared three independent studies where FUS was either knocked out or mutated to remove its nuclear localisation signal in mouse neuronal tissues.
Unlike chapter 3, these mutant mice were homozygous for their mutations.
My joint analyses combining either the knockout or mutant FUS samples together increased the detectable number of RNA regulation changes compared to analysing the datasets separately.
Overlapping the two joint models demonstrated significant similarity between the two FUS conditions in both gene expression and splicing, suggesting that FUS NLS mutations deplete FUS from the nucleus.
There were few RNA regulation changes found specifically in the mutations.
Looking in detail at the FUS locus itself, I identified a novel mechanism by which FUS could regulate its own translation through intron retention. 
This autoregulation mechanism may be complementary to one previously reported exon skipping \citep{Zhou2013}.

All together, my results suggest specific consequences for RNA regulation occuring with both loss of TDP-43 or FUS from the nucleus and from cytoplasmic mislocalisation.

\section{Issues arising}

\subsection{Loss and gain of splicing function in TDP-43 and FUS}
% review the results
% what have we learned?
% what questions are outstanding?

Both loss and gains of function are relevant to ALS/FTD because of the nuclear depletion of TDP-43 and FUS in the end-stage of the disease  is accompanied by cytoplasmic aggregates.
The formation of aggregates may be encouraged by rare mutations.
My work has attempted to understand the impact on splicing that could occur in human ALS/FTD patients.

For TDP-43, knocking down or knocking out the protein in human or mouse cells uncovered a physiological role for TDP-43 in repressing non-conserved cryptic exons. 
This was also seen in the RRM2mut mouse line where one of the RNA-recognition motifs was mutated.
However, a mutation in the low-complexity domain, where most ALS mutations are found, led to the inverse phenomenon, the skipping of constitutive exons.
These two phenomena appear to be mutually exclusive, suggesting that the LCDmut does not cause a loss of splicing function but instead an aberrant gain of function.
Although the mechanism for this has not been identified, it points to a role for TDP-43 in maintaining constitutive exon splicing.
This was reinforced by wild-type TDP-43 iCLIP binding peaks observed on and around the "skiptic" exons.

Since I began studying TDP-43 and cryptic exons, many more RNA-binding proteins have been found to have a role in cryptic exon repression including PTBP1/2, RBM17, hNRNP L and MATR3 \citep{Ling2016,Tan2016,McClory2018,Attig2018}. 
My own assessment of ENCODE knockdown data on a large number of RNA-binding proteins suggests that cryptic exon repression is a near-universal phenomenon. 
This included the majority of ALS/FTD associated proteins tested, with the exception of FUS.
Skiptic splicing is the inverse of cryptic splicing. The two phenomena represent opposite ends of aberrant splicing caused by extreme changes in protein homeostasis. The skiptic exons emerge due to rare mutations in TDP-43 and the cryptic exons through artificial (or biological) TDP-43 depletion.  Due to the tight regulation of the levels of RNA-binding proteins, these aberrant splicing events shouldn't occur within the boundaries of normal physiology. Therefore, there is an outstanding question as to when in the course of disease these events first begin, whether for use as a biomarker or as a therapeutic target. 
It would also be interesting to look for more examples of skiptic exons in other RNA-binding proteins.


%The two phenomena could represent a balance between surveillance and maintenance in RNA splicing.
%Repression of aberrant cryptic exons is a surveillance function to prevent unwanted splicing.
%Conversely the support of constitutive exons, seen in the normal state through iCLIP binding, is an maintenance function.
%This role of TDP-43 is then exacerbated by the LCDmut mutation, which leads to exon skipping. 

% balance between surveillance and maintenance?

The discovery of cryptic and skiptic splicing leads to multiple questions on the nature of cooperation between splicing factors.
% in cryptic exons - how do they emerge?
Most introns are bound by multiple splicing factors and yet a number of them are specifically mis-spliced as cryptic exons when TDP-43 is depleted.
This suggests that in certain introns TDP-43 cannot be substituted for when it is depleted. 
This is a lack of biological redundancy. 
Where these non-redundant sequences appear would depend on which splicing factors are expressed at the time.
This is shown by a cell-type specificity for particular TDP-43 cryptic exons, as seen by myself and others \citep{Jeong2017}. 
It would also suggest that each set of cryptic exons in a particular cell type are unique to each RNA-binding protein.
This could be tested with the ENCODE corsortium data.

% autoregulation
Another key mechanism to be understood in RNA biology is that of autoregulation.
Both TDP-43 and FUS proteins bind their own transcripts and modulate their own translation.
%This has made knockdown studies of these proteins difficult due to the compensation afforded by autoregulation.
However both proteins appear to do this by a complex set of mechanisms involving both NMD and nuclear detention.
The TDP-43 autoregulatory locus is the 3\'\ UTR, the site of TDP-43 binding \citep{Polymenidou2011}. 
This binding alters 3\'\ UTR length to create a nuclear detained isoform, accompanied by NMD-sensitive spliced UTR isoforms \citep{Ayala2011,Koyama2016}.
Both the RRM2mut and LCDmut mutations appear to alter TDP-43 autoregulation and this may account for the gain in splicing function seen in LCDmut.
FUS binds introns 6 and 7 \citep{Lagier-Tourenne2012}, which was previously identified to cause skipping of the central exon to create an NMD-sensitive transcript \citep{Zhou2013}
My study of the FUS introns 6 and 7 has suggested that retention of both introns creates an NMD-insensitive transcript that is potentially detained in the nucleus.
I would predict that these mechanisms play a role in ALS/FTD where either TDP-43 or FUS are mislocalised from the nucleus to the cytoplasm in affected neurons and glia.
Reduction in autoregulation would lead to a positive feedback mechanism where enhanced translation would then increase cytoplasmic TDP-43 or FUS.

% cross-regulation
Beyond autoregulation there is the intriguing phenomena of \textit{cross-regulation}, where depleting one splicing factor impacts the expression or splicing of many other splicing factors.
This has been previously seen in a study of multiple hNRNP proteins \citep{Huelga2012}. 
This study demonstrated asymmetric relationships between pairs of proteins, where depletion of one protein would upregulate another but not vice versa.
With FUS nuclear depletion I observed upregulation of the FET family members \textit{Taf15} and \textit{Ewsr1} accompanied by changes in intron retention seen in both transcripts. 
This suggests a feedback mechanism where reduction in nuclear FUS leads to a direct increase in FET family levels, potentially to compensate for reduced nuclear FUS due to the overlapping motifs and targets of the three proteins, although TAF15 knockdown fails to alter \textit{FUS} gene expression \citep{Kapeli2016}.
Many other splicing factor genes were upregulated and alternatively spliced in FUS nuclear depletion, suggesting a generalised response of the splicing machinery and associated factors to loss of one particular protein. 
This phenomena must depend on nuclear FUS levels, as this was only seen in the homozygous NLS mutant mice and not in the FUS $\Delta$14 adult mice studied in chapter 4.
There I found no large changes in splicing factor gene expression and splicing except for \textit{Fus} and \textit{Ewsr1}.
It has been previously observed that TDP-43 binds the same region of the \textit{FUS} transcript that FUS binds itself, linking the two proteins \citep{Lagier-Tourenne2012}.
It would be interesting to look at changes in the putative FUS autoregulatory transcript in my collection of TDP-43 depletion and mutation mice.
% this is called cross-regulation. Extend this. 

% low-complexity domains
The TDP-43 LCDmut mouse was created to study the effect of mutation on the low-complexity domain. 
It is striking that low-complexity domains are found in almost all RNA-binding proteins associated with ALS/FTD.
In TDP-43 most disease-associated mutations are found there.
In disease, the low complexity domains have been implicated in protein aggregation and stress granule formation but it still unclear what the physiological role of these domains are.
One hypothesis is that these domains allow the assembly of multiple RNA-binding proteins on a transcript \citep{Gueroussov2017}. 
This would allow splicing factors to be brought in contact with a transcript that lacks known sequence motifs and would answer why many altered splicing events seen in my work have no nearby CLIP binding peaks.


% RNA biology questions:

% How do splicing factors cooperate? 
% Cryptic exons probably represent parts of the genome where there is less redundancy, only one factor can fully repress them.

% How changing the level of one factor effects the levels of others.
% FUS nuclear depletion causes upregulation in fellow FET family members TAF15 and EWSR1. 
% It also leads to changes in expression and splicing of many other splicing factors.

% What are the role of low-complexity domains in splicing factor interactions


%Unlike cryptic exons, constitutive exons are well-studied and defects in constitutive exon recogntion of the same exons by multiple factors may be more apparent.
%Further work on this may break down the current hard divisions between "constitutive" and "alternative" exons. 
%This will require more work looking at "constititutive" splicing across mutiple tissues, such as the GTEX consortium. 





% TDP loss leads to de-repression of cryptic exons
% these exons are not conserved between human and mouse - presumably due to lack of conservation of introns
% many alternate exons regulated by TDP are not conserved either or differ greatly in their change in response to TDP between human and mouse
% even within the same species they do not completely overlap between datasets
% each dataset has unique sets of exons
% This was seen in another study comparing TDP-43 deletion in mouse neurons, muscle cells and stem cells \citep{Jeong2017}.
% Each cell type will have a different set of RBPs expressed and binding to the same introns.
% the two human K562 cell datasets from ENCODE do not fully overlap despite being the same cell type
% some of this is probably technical due to polyA+ vs Total RNA
% but even the smaller list of total RNA cryptic exons are not merely a subset of the larger polyA + exons


% how important is loss of function to disease?

% multiple mouse models have shown that neurodegeneration can occur both without loss of TDP or FUS splicing function nor cytoplasmic aggregates.
% Mice are not humans

% Intron retention and complex splicing have been under-studied. 
% In chapter 5 I focussed on cassette exons due to the tractability of working with these splicing events. 
% But lots of information is lost.
% Complex events are subtle and hard to classify and may therefore be more prone to false positives.
% Intron retention is also highly dependent on library preparation. 
% In chapter 6 the polyA+ enriched FUS data (Dupuis) had very few detectable intron retention events compared to the other two total RNA datasets but this is also confounded by read length and depth.
% There may however be interesting biology in comparing intron retention via polyA+ and total RNA as the latter measures pre-mRNA as well as mature mRNA.
% Changes in total RNA intron retention may represent co-transcriptional changes whereas intron retention detected in polyA+ could be post-transcriptional splicing dynamics. 
% The FUS intron retention events are observed in both total and polyA+ libraries so is suggestive of the latter.

% RNA-binding proteins do not act alone.



\subsection{RNA-seq: the value of replication and annotation-free analysis}

RNA-seq allows phenomena previously seen in individual genes to be demonstrated genome-wide.
In the example of TDP-43 and splicing, initial studies on individual cassette exons have given way to hundreds of splicing events being found across different tissues and developmental time points.
The creation of RNA maps allow general principles of a protein's effect on splicing to be demonstrated and clearly visualised.
We can build up complete pictures of what an RNA-binding protein can do.

With each RNA-seq study, new phenomena and mechanisms can be identified.
Crucially, as each published dataset is available for re-analysis, these phenomena can be replicated in multiple species and tissues.
By combining multiple published datasets in chapter 3 I have strengthened the case for TDP-43 and cryptic exons.
The novel mechanism of FUS autoregulation found in chapter 6 is much stronger for having been observed in 3 independent studies.
Replication of RNA phenomena is crucial as there is so much variability between experiments caused by RNA extraction, library preparation, and analysis software.
With so many findings arising from individual RNA-seq experiments in a particular species or cell type there is a real need for these to be replicated.

Annotation initiatives have been successful in cataloguing genes and transcripts in many species.
They are very useful for any scientist interested in RNA biology by providing the genomic coordinates for a set of experimentally-validated splicing isoforms.
However, there is great value in looking outside what is given to you by annotation.
In under-studied cell types like neurons there is a wealth of unannotated splicing in normal cells.
When perturbing splicing one could be expect there to be unusual or aberrant splicing that would not normally be observed. 

Both cryptic exons and skiptic exons are examples of novel splicing events that are not detected when relying on annotated transcripts alone.
I anticipate more developments in the RNA-seq field on tools that discover and quantify novel splicing.
Beyond this, there is a pressing need to verify novel RNA isoforms at the protein level, if they are indeed translated. 
This will require new computational tools for shotgun mass spectrometry.

	
\section{Future directions}

\subsection{Translating work into human patients}

It is regrettable that throughout my PhD I have not had the opportunity to explore more human RNA-seq samples, particularly those from human ALS/FTD patients.
This is partly due to the inherent variablity between human subjects but also because of the technical hurdles in analysis.
Unlike laboratory strains of mice, humans are highly heterogenous in both their genetics and the environmental exposures.
Even patients with identical mutations in a disease-associated gene could  expected to have many distinct modifying genetic variants as well as completely different life experiences.
Therefore any case-control study on transcriptomics in humans requires a much larger sample sizes to account for the increased variability in measurements of gene expression and splicing.
Alternatively, this genetic variability can be exploited by a study design that correlates genetic variation with other features. 
Genome-wide association studies (GWAS) correlate common or rare genetic variation with clinical phenotypes.
The latest GWAS on ALS combines nearly 100,000 samples \citep{Nicolas2018}.
Gene expression and splicing phenotypes from RNA-seq can be correlated with genetic variation to produce lists of genetic variants associated with specific changes in RNA regulation.
These genetic variants can then also be associated with clinical phenotypes, as found in a large Alzheimer's disease cohort \citep{Raj2018}.

There are also technical considerations in sequencing post-mortem brains.
Both diseased and control donor brains will often come from human patients who were kept on artificial respiration before death. 
This is known to affect RNA quality in post-mortem tissue \citep{Durrenberger2010}.
Most post-mortem tissue RNA-seq samples are prepared using total RNA libraries to avoid bias towards the 3\'\ end that comes with polyA+ purification of degraded RNA.
Total RNA libraries will dilute the amount of reads that align to protein-coding genes, requiring higher coverage and greater cost per sample.
There are now large consortia established to collect and sequence both RNA and DNA from large numbers of patient tissues. 
As part of my post-doctoral studies I am looking forward to working with these samples.

An alternative route for studying ALS/FTD in humans is to grow patient neurons from induced pluripotent stem cells (iPSCs).
These can be created from skin samples from patients which are reverted a pluripotent state and then differentiated into disease-specific cell types such as motor neurons.
What is particularly powerful is that patient cell lines can be edited to remove the causative mutation, creating an isogenic control line.
This effectively controls for human genetic variability, and so any phenotypes can be directly associated to the effect of the initial mutation.
Studies comparing iPSC-derived motor neurons are providing powerful insights into changes in RNA regulation within motor neurons, as seen for microRNAs in FUS \citep{DeSantis2017} and intron retention in VCP \citep{Luisier2018}. 

\subsection{Diving deeper into RNA regulation}

% bulk cell sequencing - need single cells
All the work in my PhD was done with bulk-cell short-read RNA-seq. 
There are limitations to this approach that are both biological and technical.
Bulk RNA sequencing samples RNA from all cells in a population. 
This limits the observations that can be made to changes in expression and splicing that can be seen across a wide range of cells.
It also demands low variability between cell type mixtures between conditions to avoid this confounding results.
A rapidly developing solution to this hurdle is single-cell RNA sequencing, where individual cells are first separated from a tissue and sequencing libraries prepared using microfluidic technology.
Studies are now emerging of 10s of 1000s of individual cells being profiled. 
This has allowed the entire mouse nervous system to be classified into different cell types \citep{Zeisel2018}. 
The technology for this is still in its infancy and there is a problem with sequencing throughput to get enough reads per sample to robustly measure splicing.
It would be very exciting if this technology could be applied to ALS/FTD patient brains. 

% need specific populations of RNA
Bulk sequencing of cells also removes the distinction of RNA localisation within different cellular compartments. %distinction wrong word
For example, both FUS and TDP-43 bind transcripts in the cytoplasm and have some role in axonal transport \citep{Fallini2012,Fujii2005}. %ref
Separating RNA into nuclear or cytoplasmic fractions would give information about nuclear retention of particular transcript, as I hypothesise for the FUS intron retention transcript identified in chapter 6.
This can be done biochemically by fractionation, or physically by the use of growing neuronal cells in compartments to physically isolate RNA \citep{Taliaferro2016}.
Alternatively, sequencing the nascently transcribed RNA would allow observation of unstable transcripts.
The nagging question from chapter 3 is that if I predict most cryptic exons to be degraded by NMD, why do I see any at all?
If we were to sequence nascently transcribed RNA in TDP-43 depletion and controls we could identify unstable transcripts that rapidly degrade upon export to the cytoplasm.
This may demonstrate cryptic splicing repression by TDP-43 to be much more widespread and why certain cryptic exon transcripts appear to evade NMD.

% long reads
The other hurdle with the current state of RNA research is the length of sequencing reads.
The current gold standard for Illumina sequencing is paired end 300bp reads.
This means that splicing analysis requires the reconstruction of complex isoforms from individual short fragments.
Long-read sequencing techniques such as Pacific Biosciences and Oxford Nanopore promise to sequence entire transcripts in a single read.
This would greatly simplify analysis but are still developing. 
Both techniques suffer from a high error rate which will complicate read alignment. 
Throughput is also an issue. It is not yet known what sequencing depth is sufficient to analyse alternate splicing with long reads.
Oxford Nanopore technology can be used to sequence native RNA rather than cDNA, which allows the observation of RNA modifications.
A compromise between long and short reads are synthetic long reads, where short Illumina reads are generated from an individual transcript suspended in an oil drop. 
This has allowed the reconstruction of individual transcripts and the curious finding that some splicing events are co-regulated within a transcript \citep{Tilgner2015}.

As with the emergence of RNA-seq 10 years ago, technical breakthroughs in RNA biology will uncover new and exciting types of RNA regulation.
I hope these developments will enhance our understanding of the molecular mechanisms of ALS/FTD.

% problem with bulk sequencing
% lose cell-type specific information
%% single-cell RNA seq can do this
%% in disease brain one could select just cells with TDP or FUS pathology for analysis
% lose RNA localisation information
% some isoforms are preferentially transported - see Taliaferro
% Cytoplasmic NLS mutant FUS will play a role in transport - would need to fractionate

%Bulk cell or whole tissue sequencing means a loss of resolution
%RNA extracted from entire sections of tissue or populations of cells
% problem with short read sequencing
% cannot easily resolve complex transcripts
% evidence that some splicing events are co-regulated from Tilgner et al, need long reads to see this
% intron retention would be much simpler to detect
% still issues with throughput and error rate

%
%
%\section{Final thoughts}
%
%% FINISH
%% the RNA hypothesis 
%The RNA hypothesis of ALS/FTD has been an attempt to marry together lines of evidence from genetics and pathology into a model to explain the disease. 
%Like all models, it is an oversimplification.
%
%New technology promises to expand our understanding of RNA regulation.
%The future is bright for the fields of RNA biology and neurodegenerative disease. 
%
%% TDP but not FUS represses cryptic exons
% However cryptic exon repression can be seen in many RNA-binding proteins
% Repression of intronic sequences will be achieved by multiple proteins simultaneously
% "Crypic exons" that we see are just the tip of an iceberg, those sites where a single protein does most of the work
% Would explain low overlap between different knockouts in the same species
% Follow up paper showed lack of overlap between different mouse tissues

% TDP LCD mutations cause a gain of splicing function
% Inverse to cryptic exons, this is a role in maintaining canonical splicing that is being altered somehow by the mutation
% Suggests a spectrum between surveillance and maintenance functions
% Canonical splicing must be maintained and aberrant splicing must be repressed.
% TDP has roles in both
% It would be interesting to look through a range of RBPs and assess their skiptic and cryptic contributions - do they fit on a spectrum?
% Might expect proteins more deeply involved with the spliceosome to be more involved in supporting canonical exons than repressing
% Although one could also expect both functions depending on the position of binding, as are seen for alternate exons.

% FUS diverges from TDP, despite having shared roles in long intron genes, RNA transport and stress granules
% FUS does not repress cryptic exons, this could be due to weaker motif
% FUS KO and NLS mutations overlap, essentially both are loss of FUS function
% Splicing changes show few cassette exons and these are not enriched in iCLIP peaks.
% Excess of intron retention and complex events potentially an artefact of total RNA libraries, although the Dupuis dataset is polyA+ and contributes somewhat to splicing although less than the others.

% All the high quality mouse TDP data is polyA+ enriched so cannot make a fair comparison with the FUS knockout data.
% This loss of function is not seen in studying the delta 14 mice, as very few splicing events are seen.
% Gene expression profiles are completely different between embryonic HOM D14 and 12 month HET D14.
% The former shows neuronal and RNA binding terms, probably due to acute FUS loss
% Mitochondrial and ribosomal changes seen in 12 month D14 could be due to gain of function
% Other observations of FUS granules in the paper right?




% What does this mean for studying TDP and FUS?
% Two proteins have different functions
% They diverge on crypic exons



% What does this mean for ALS/FTD?
% Loss of RBP function is only one facet and it will be important to untangle these effects from a gain of function. The LCDmut shows that  without aggregation there is a specific effect on splicing, which can also be seen in human ALS patients


% What does this mean for RNA biology?



% What kind of work will follow this?










%\chapter{Next Steps for my PhD}
%
%My work so far has been on two very different projects centering around RNA regulation in models of ALS/FTD. The cryptic splicing project in TDP-43 was based on public data, which allowed for rapid progress and the ability to replicate my conclusions across multiple experiments. However, brute force knockdown of TDP-43 is a very crude model of disease and the project suffered from the low quality of RNA-seq data and the inability to validate the mechanism with a wet lab collaborator. The FUS $\Delta$14 mouse project however offers up a humanised mouse model where disease-like changes occur progressively. There is much appetite from Anny Devoy and from the lab of Pietro Fratta to continue investigations into the mechanism of the mutation's effect on RNA regulation.\\ 
%
%Over the remainder of my PhD I plan to upgrade the RNA-seq pipeline to use state-of-the-art splicing analysis software. I will then apply this pipeline to better understand the changes in splicing that occur in the FUS $\Delta$14 mouse, as well as in new data being generated on the mutation's impact on mRNA transport and translation. I will also apply the splicing pipeline to two opposing \textit{TARDBP} mutant mice and compare the results with existing TDP-43 knockdown data. To continue the cryptic splicing project I will screen the large ENCODE cohort of RNA-binding proteins for cryptic exon repression and then look for evidence of cryptic splicing in ALS/FTD patient brains. 
%	
%
%\section{Developing a full-featured splicing analysis pipeline}
%Since I started work on \textit{CryptEx} in October 2015 there have several major advancements in algorithms to analyse differential splicing from RNA-seq data. This has been enabled by improvement in the quality of RNA-seq data. \textit{CryptEx} and the underlying \textit{DEXSeq} approach were designed to work on all qualities of data going back to low depth, short read datasets generated in 2010. So while it has been useful for the TDP-43 project I believe I should make use of the state-of-the-art tools now available. Dr Kitty Lo and I have installed and assessed the latest advances in RNA-seq analysis algorithms. We can group software packages into one of four categories based on what they consider to be the fundamental unit of differential splicing. All of these modern algorithms require paired end data with high coverage and long reads to maximise the information available.
%
%\subsection{Transcript quantifiers}
%These compare abundances at the transcript level and rely on a curated list of known transcripts. Sequencing reads are then probabilistally assigned to a each transcript. This can be done extremely quickly in the case of pseudoaligners like \textit{Kallisto} \citep{Bray2015}. Transcript assembly algorithms like \textit{Stringtie} \citep{Pertea2015} can even assemble reads into novel transcripts. These algorithms require long reads for enough splicing information and very dependent on the list of transcripts they are given. Some genes can generate hundreds of very similar transcript isoforms or overlap with other genes and so these methods will struggle to accurately assign reads to them.
%
%\subsection{Exon quantifiers}
%These algorithms are typified by the \textit{DEXSeq} approach \citep{Anders2012}, where the transcripts are flattened to create a set of unique exon "bins".  As the bins themselves may not actually be an exon it makes the results difficult to interpret, although this has been lessened in a new software package called \textit{JunctionSeq} \citep{Hartley2015}, which also quantifies splice junctions and can pick up novel splice junctions. As they consider the whole gene they are sensitive to poorly annotated genes and extreme biases in coverage.
%
%\subsection{Local splicing event quantifiers}
%These algorithms work by comparing differences in splice junctions across the body of a gene while being agnostic to transcript annotations, allowing for the detection of novel events. These algorithms then establish different categories of possible events such as cassette exons, exon extensions, intron retention and alternate first or last exons. Each event is then tested for differences in inclusion/exclusion across conditions. This approach has been implemented in the \textit{SUPPA} \citep{Alamancos2015}, \textit{SGSeq} \citep{Goldstein2016} and \textit{MAJIQ} \citep{Vaquero-Garcia2016} packages. These packages, while providing a clear output of different splicing events, struggle to determine splicing events such as tandem 3'UTR changes which can only be differentiated by read coverage and not splice junctions. 
%
%\subsection{Annotation-free quantifiers}
%The fourth and final class of algorithm don't depend on any transcript annotation at all. The \textit{derFinder} package \citep{ColladoTorres2015} assesses changes in read coverage between samples, which is particularly useful in looking for novel non-coding RNA species or in the aforementioned tandem 3'UTR problem. The \textit{LeafCutter} package \citep{Li2016} looks for differentially used splice junctions independently of annotation and has shown that across a set of tissues around 30\% of the high confidence splice junctions recovered are novel to any annotation database.\\
%
%Dr Lo and I will create a pipeline that implements a combination of the above approaches to gain a full picture of splicing dysregulation in the FUS $\Delta$14 mouse and also in a different set of TDP-43 mutant mice.
%
%
%%%
%%%Stringtie is the successor to Cufflinks for assembling transcripts \citep{Pertea2015}.
%%%
%%%The DEXSeq methodology suffers from discarding the valuable information given by splice junctions and that has now been incorporated into the JunctionSeq package . 
%%%Categorising splicing changes
%%%SGSeq - what good for? \citep{Goldstein2016}
%%%SUPPA \citep{Alamancos2015}
%%%Annotation-free quantification of splicing derFinder \citep{ColladoTorres2015}
%%%LeafCutter \citep{Li2016}
%%%
%%%In collaboration with Dr Kitty Lo, we will build a state-of-the-art splicing pipeline.
%
%\section{RNA dysregulation in two opposing \textit{TARDBP} mutant mice}
%
%The lab of Dr Pietro Fratta have generated two mouse lines with mutations in TDP-43 from a large mutagenesis screen \citep{Ricketts2014}. One line, F210I, contains a mutation in the second RNA recognition motif and another, M323K contains a mutation in the C-terminal low complexity domain which is where the majority of ALS patient mutations occur \citep{Sreedharan2008-xv}. The F210I mutation is embryonic lethal in homozygosity whereas the M323K mutation appears to lead to a progressive motor neuron loss in adult mice. The Fratta lab has generated high quality RNA-seq data for the embryonic F210I and the adult M323K mice. Kitty Lo and I will apply our state-of-the-art splicing pipeline to investigate the splicing changes and compare them to existing TDP-43 knockdown data used in the cryptic splicing project. 
%Preliminary analysis has demonstrated a large number of cryptic exons in the F210I embryonic mice that overlap with the TDP-43 knockdown data from \citep{Polymenidou2011}. The lab has also generated iCLIP data by precipitating the RNA targets of the two mutant proteins (Dr Agnieszka Ule and Prasanth Sivakumar). It will be very interested to see how the two mutations alter TDP-43's splicing functionality and how that relates to toxicity and neurodegeneration.
%
%\section{Dissecting the role of FUS in mRNA transport and translation}
%The gene expression analysis of the FUS $\Delta$14 mice showed a general downregulation of ribosomal mRNAs. This is interesting as FUS has been noted to play a role in mRNA transport outside of the nucleus for local translation \citep{Kanai2004,Fujii2005} as well as modulating translation by transporting RNA to translationally active granules \citep{Yasuda2013} or to translationally silent stress granules \citep{Bosco2010}. In the light of this, the lab of Pietro Fratta want to observe the effect of the $\Delta$14 FUS mutation on both RNA transport in axons and the modulation of translation. 
%The impact of the FUS $\Delta$14 mutation on axonal transport will be observed by performing RNA-seq on spatially separated samples of motor neuron cell bodies and axons. This approach has been successful in wildtype neuronal cells \citep{Taliaferro2016}, where specific splicing isoforms were enriched in axons compared to cell bodies. This work will be carried out by Dr Nicol Birsa. I will process and analyse RNA-seq libraries created from both the axons and the cell bodies of wildtype and FUS $\Delta14$ motor neurons. I will try to compare axon-enriched mRNAs between wildtype and disease to uncover which mRNAs are affected by the FUS mutation, either by changes in localisation or changes in the splicing isoform for that gene.
%
%Polysome profiling involves immuno-tagging of a specific ribosomal protein in a specific cell type. These tagged ribosomes can be pulled down and the bound RNA can be sequenced to determine the RNA species and isoform \citep{Shigeoka2016}. This can then be compared to a background sequencing library to demonstrate an enrichment of a set of RNAs bound to ribosomes in mouse motor neurons carrying either the wildtype or mutant FUS. This work will be carried out by Dr Agnieszka Ule. I will then analyse RNA-seq libraries created from the ribosome-bound RNA to look for changes in translation between the wildtype and FUS $\Delta14$ motor neurons.
%
%By combining the two datasets I should be able to build a model of how the FUS $\Delta$14 mutations affects both the transport and translation of mRNAs. 
%
%\section{Screening the \textit{ENCODE} RNA-seq cohort for cryptic splicing}
%In chapter 3 I made use of publicly available RNA-seq data generated by the \textit{ENCODE} consortium (\url{www.encodeproject.com}) which has performed shRNA knockdown experiments on over 150 RNA-binding proteins, including the ALS/FTD-linked proteins TDP-43 and FUS, as well as MATR3, hnRNPA1, hnRNPA2B1 and TAF15. My cryptic splicing project concluded that TDP-43 but not FUS plays a role in repressing nonconserved elements of the genome from being included into transcripts but whether or not this occurs in other ALS-linked genes is unknown. There has not been a large screen of cryptic splicing across multiple proteins as previous studies have focussed on individual candidates: hnRNP C \citep{Zarnack2013-nv}, PTBP1/2 \citep{Ling2016} and RBM17 \citep{Tan07102016}.
%I therefore plan to download and align all shRNA knockdown data from ENCODE and run the data through both my own cryptic splicing pipeline (\textit{Cryptex}) and through a number of different algorithms that quantify novel splicing. I have already processed the data from the previously mentioned proteins and compared the number of high confidence cryptic exons discovered by \textit{Cryptex} with the strength of the knockdown of the target gene calculated by \textit{DESeq2} \citep{Love2014} (Figure 5.1). The results are promising as it appears that MATR3, TAF15 and hnRNPA2B1 knockdown leads to inclusion of a large number of cryptic exons, but only in experiments with a sufficiently strong knockdown. The results illustrate the variability in shRNA knockdown throughout the cohort. In TDP-43 around 70 cryptic exons can be seen in the K562 leukaemia cell line but fewer than 10 can be observed in the HepG2 liver carcinoma cell line, which has a much poorer knockdown. Screening the ENCODE cohort for cryptic splicing may uncover a new set of cryptic exon repressing RNA-binding proteins, or may even show that cryptic splicing repression is a common function of many RNA-binding proteins. 
%
%\begin{figure}
%	\begin{center}
%		\includegraphics[width=12cm]{Figures/misc/ENCODE_summary.png}
%	\end{center}
%	\caption{ALS and cryptic splicing linked protein knockdowns from the ENCODE consortium  }
%\end{figure} 
%
%\section{Cryptic splicing in ALS and FTD patient brain}
%The initial paper on TDP-43 and cryptic splicing observed that two cryptic exons out of the 50 observed in HeLa cells could also be observed in the brains of ALS patients but not in non-neurological disease controls \citep{Ling2015}. Since then no further work has studied cryptic splicing in ALS or FTD brains. There are multiple lines of evidence that suggest that a number of RNA-binding proteins are perturbed in ALS/FTD as well as TDP-43 and FUS. Therefore if all the potential cryptic targets were known for a host of RNA binding proteins, the inclusion of these cryptic exons in diseased brain would suggest a perturbation in expression of that protein. 
%Building on the results of the TDP-43 cryptic splicing project and the results of the ENCODE screen, I want to thoroughly search for evidence of cryptic splicing in a set of control and disease brains. One such set of sporadic and familial ALS brain RNA-seq data has been previously published \citep{Prudencio2015}. We also have an in-house set of FTD and control brain RNA-seq (unpublished). The data will need to be very carefully prepared. I will attempt to deconvolute gene expression by cell type to compare the levels of neurons and glial cells in the control and diseased brains, to better understand differences in splicing changes between disease samples. RNA-seq of purified cells is available in mouse \citep{Newman2016} and algorithms for this specific task have been developed \citep{Zhang2014}. This approach has successfully applied to human brains to investigate cell type changes in aging \citep{Soreq2017}. There is also scope to look in patient blood and cerebrospinal fluid for potential RNA biomarkers to track disease course. 
%
%\section{Conclusions}
%My work so far has investigated RNA regulation in models of both sporadic and genetic ALS/FTD. Completion of my PhD should result in a fuller picture of cryptic splicing as a general mechanism of disease, complemented by a fine-grained analysis of specific mutations in \textit{TARDBP} and \textit{FUS}. Taking these insights into human patient tissue will be challenging but may lead to advances in understanding both ALS and FTD. 
%
