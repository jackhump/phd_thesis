% Two different strands here - Disease biology (the problem) 
% and sequencing technology (the method) to generate insight

\chapter{Introduction}

\section{Amyotrophic Lateral Sclerosis and Frontotemporal Dementia} % the disease problem

%introduce two diseases, compare and contrast
Amyotrophic Lateral Sclerosis (ALS) is a progressive neurodegenerative disorder primarily affecting the motor neurons of the cerebral cortex and the spinal cord. It affects 2-16 people per 100,000 \citep{Logroscino2010}. Patients gradually lose voluntary motor control of their limbs and the muscles involved in speaking and swallowing. Death usually occurs within 2-3 years after the first sign of symptoms, usually from infection caused by the inability to swallow. Frontotemporal Dementia is a progressive neurodegenerative disorder primarily affecting the frontal and temporal lobes. It affects 15-22 people per 100,000 and is the second most common dementia after Alzheimer's disease \citep{Onyike2013}. Depending on the subtype of FTD, patients exhibit worsening behaviour inhibition, language production or comprehension. Both disorders peak in incidence at around 60 years of age, are invariably fatal and have no cure. These two disorders are now recognised to be two ends of a continuum called ALS/FTD. This is in part due to a sharing of symptoms in some cases, as FTD patients can exhibit motor deficits and ALS patients can exhibit cognitive decline, but also due to a striking concordance in pathology and genetics. 

\subsection{Both disorders share pathology}
% pathology work - identification of TDP-43 and FUS inclusions

Both disorders have recognisable brain pathology upon autopsy, with the affected brain regions showing aggregated protein inclusions in the nucleus and cytoplasm of neurons and glia. In FTD around 35\% of patients have inclusions positive for Tau, a microtubule-associated protein encoded by the \textit{MAPT} gene also linked to Parkinson's and Alzheimer's disease \citep{Rademakers2004}. The rest of FTD patients present with ubiquitinated inclusions containing one of two proteins: TAR DNA-binding protein 43kDa (TDP-43) \citep{Neumann2006-re} or fused in sarcoma (FUS) \citep{Neumann2009}. In ALS almost all patients present with TDP-43 positive inclusions \citep{Neumann2006-re} and a small number display FUS inclusions \citep{Vance2009-ye}, firmly cementing the link between the two disorders and a key role for TDP-43 and FUS.

\subsection{Both disorders share genetics}
% Genetics - following SOD1, the identification of rare TARDBP and FUS mutations.  
The progress in understanding the pathology of ALS/FTD has been mirrored by the progress in locating causative genes. This was initially done by linkage studies, where blocks of shared genetic information were identified in the affected members of large families. \textit{SOD1} was the first gene linked to ALS in series of families over 20 years ago \citep{Rosen1993}. Mutations in MAPT were then found in a large number of familial FTD cases \citep{Hutton1998}, linking the protein pathology with alterations to the gene itself. This theme continued in the discovery a series of rare mutations in TARDBP, the gene that codes for TDP-43 in familial cohorts of ALS and FTD \citep{Sreedharan2008-xv}. This was followed by the discovery of patients carrying mutations in the FUS gene \citep{Vance2009-ye}. The last hurrah for linkage studies came in the solving of a long standing mystery. Multiple ALS and FTD pedigrees had been linked to a region on chromosome 9, which was revealed to be a large expansion in the intron of the \textit{C9orf72} gene \citep{Renton2011,DeJesus-Hernandez2011}. In individuals of caucasian ancestry the expansion is found in 5-10\% of sporadic ALS and FTD cases, 40\% of ALS and 25\% of FTD cases with a family history \citep{Majounie2012}, more than all the other known genes put together and making it the single largest genetic contribution to ALS/FTD. At the same time, the emergence of next-generation sequencing technologies has moved the gene hunting field from conducting linkage in family pedigrees to large-scale studies comparing the allele frequencies between groups of affected and unaffected people, at first in exomes (the total protein coding portion of the genome) and soon to full genomes. This has been extremely fruitful in identifying causative mutations in a wide range of genes, recently reviewed in ALS \citep{Taylor2016} and FTD \citep{Pottier2016}. Broadly, the proteins these genes code for can be grouped by their functions. \textit{OPTN}, \textit{UBQLN2}, \textit{SQSTM1}, \textit{CHMP2B} and \textit{TBK1} have all been linked to protein degradation, whereas \textit{DCTN1}, \textit{CHCHD10} and \textit{TUBA4A} have been linked to microtubule transport and stability. The third group of genes encode RNA-binding proteins, and this is the function of \textit{TARDBP} and \textit{FUS}, as well as \textit{MATR3}, \textit{TAF15}, \textit{hnRNPA1} and \textit{hnRNPA2B1} . The proteins these genes code for have been linked to splicing, transcription, translation and transport of mRNA. 

The evidence from both the pathology and the genetics together create the RNA hypothesis of ALS and FTD, where impaired RNA regulation due to mutations or mislocalisation of RNA binding proteins is progressively toxic to neurons.

%Advances in high throughput sequencing technology have enabled the finding of causative genes in both disorders, from small family pedigrees to large international cohorts. Unsurprisingly, rare mutations have been found in \textit{TARDBP} the gene coding for TDP-43 \citep{Sreedharan2008-xv}, as well as in \textit{FUS} \citep{Vance2009-ye}.
%
%
%
%
%Each year more genes are linked to ALS and FTD and a sizeable proportion of the proteins they code for bind RNA. These RNA-binding proteins  
%%RNA binding proteins MATR3, TAF15, hnRNPA1, hnRNPA2B1
%%
%%FTD has incidence of 15-22 per 100,000 \citep{Onyike2013}
%%
%%ALS has incidence of 2-16 people per 100,000 in Europe \citep{Logroscino2010}
%%

%Introduce the two diseases. Introduce the RNA regulation theory of disease but also discuss the other possibilities

%Two bad neurodegenerative diseases
%Describe clinical presentations of both
%The ALS/FTD spectrum - linked through TDP-43 and FUS
%
%Evidence for RNA processing involvement in ALS/FTD
%
%
%%ALS/FTD and RNA-binding proteins


%In the case of TDP. FUS, A1/A2, Matr3, RBPs are mutated but with C9 RBPs are sequestered by the repeat. And yet all patients have similar phenotype.


\section{Neuronal regulation of mRNA} % the biological problem
% from the paper:
%explain RNA regulation
%
%What do neurons need?
%Long genes with complex isoforms to increase protein diversity
%Need to transport RNA and protein over long distnace
%Local translation at synapses - control over protein expression

Of all the cells in the human body, neurons arguably make the largest demands upon the transcription and splicing machinery. Neuron-specific genes tend to be much longer than in other tissues \citep{Sibley2015} and an individual neuronal gene can be processed by alternate splicing to create 1000s of mRNA and subsequent protein isoforms \citep{Treutlein2014}. The distinct compartments of a neuron's architecture requires exquisite fine-tuning of protein function to suit its location, for example on either side of a synapse. There is also the matter of transport. Motor neurons can have axons over a metre long, along which an mRNA would have to travel to reach ribosomes close to a synapse for local translation. It is easy to hypothesise how small defects in splicing efficacy or mRNA transport could have catastrophic consequences for particular groups of neurons. 

%Explain splicing at a high level. The need for a high fidelity across the transcriptome and the potential for diversity in the proteome. 


\section{RNA-sequencing is a revolutionary technology to quantify RNA expression and splicing} % the new hot method

RNA sequencing, henceworth written as RNA-seq, is the application of modern high throughput sequencing to  to directly determine the sequence of input RNA molecules \citep{Wang2009}. Unlike the older microarray technology which relies on choosing a set of RNA probes to measure, RNA-seq is hypothesis-free. It is also highly sensitive and can pick up very lowly expressed genes. Instead of measuring the intensity of a probe, the abundance of a particular RNA molecule is calculated simply by counting the number of sequencing reads that contain its sequence. As sequencing technology has improved and reduced in cost, more complicated aspects of RNA regulation are now observable. Alternate splicing can be measured by the number of sequencing reads split across multiple exons: the splice junction. Complicated isoforms can be reconstructed from splice junctions where sequencing is sufficiently deep.

Multiple groups across the world have used RNA-seq to investigate gene expression and splicing changes in all manner of models of ALS/FTD. The field as a whole is progressing towards more subtle and genuinely disease-like models; from the first knock-out mice, to overexpression of human mutant proteins to the more modern knock-in models where the mutant protein is at last expressed at a physiological level. While we can be more confident that the read-outs from these experiments are closer to the the truth in humans, they suffer from a much longer generation time due to replicating diseases of old age. Changes in RNA may be subtler than we currently have the power to detect or more baroque than we can yet understand. However the real power of an RNA-seq experiment is that it is open platform. This means once the data is generated it can be used in light of whatever the most up-to-date reference genome, transcript annotations or hot hypothesis happens to be. Coupled with the requirement to make raw sequencing data publicly available, RNA-seq allows for large scale re-analyis and meta-analysis in light of new discoveries and ideas. This ease of replicability is a triumph for modern biology.

%
%Overview of RNA-sequencing and the bioinformatic tools used to assay the data
%
%Differential expression 
%
%With higher read depth and longer reads comes the reconstruction of transcripts through splice junctions
%
%Observation of splicing regulation through the creation of splice junctions



%
%\section{Genetic models of ALS-FTD} % what has been done before
%% trend towards subtler modelling of disease coupled with better quality sequencing to look for subtler effects on RNA processing
%Each chapter will discuss previous models and summarise what has been found so far.
%
%Lots of previous work on ALS-FTD linked RNA-binding proteins
%
%High level overview - models are becoming more representative of sporadic disease to unpick this
%
%
%
%

% TDP-43 

\section{TDP-43}
TDP-43 is a ubiquitously expressed RNA and DNA-binding protein encoded by the \emph{TARDBP} gene. Loss of TDP-43 from the nucleus accompanied by TDP-43 positive inclusions in the cytoplasm of cortical and spinal cord neurons is the hallmark pathology of amyotrophic lateral sclerosis (ALS) as well as the majority of  cases of frontotemporal dementia (FTD) \citep{Neumann2006-re} and some cases of Alzheimer's disease \citep{LaClair2016} . In addition, missense mutations in \emph{TARDBP} have been shown to cause familial ALS \citep{Sreedharan2008-xv}. These findings point to a central role of TDP-43 in the aetiology of ALS/FTD. 


\subsection{TDP-43 and splicing}
Previous work on TDP-43 has been focused on its role in mRNA splicing. TDP-43 was first shown to repress the inclusion of exon 9 in the \emph{CFTR} gene by binding to long UG-rich sequences \citep{Buratti2001-et}, a highly specific motif validated by X-ray crystallography \citep{Lukavsky2013}. TDP-43 was initially shown to act as both an enhancer and repressor of exon inclusion in a small number of genes \citep{Mercado2005-js,Bose2008-du,Shiga2012-it} but later work used RNA-seq and splicing-sensitive microarrays to expand the number of TDP-43 linked splicing events to hundreds  \citep{Polymenidou2011,Tollervey2011}. Both studies used RNA-protein interaction mapping (see methods) to demonstrate that TDP-43 binds on top of or close to exons which it represses and further away from the exons it enhances. However, attempts to correlate TDP-43's effect on the splicing of a gene with a change in the level of the subsequent protein has so far proved unsatisfactory with very few convincing candidates \citep{DeConti2015,Stalekar2015}.
Other roles in RNA regulation have been proposed for TDP-43, such as RNA transport \citep{Alami2013}, translation \citep{Freibaum2010-hw} and microRNA creation \citep{Kawahara2012} with a still unexplained finding of TDP-43 binding sites enriched in very long intronic sequences. Long intron genes are mostly brain specific in their expression patterns \citep{Sibley2015} and are strongly downregulated under TDP-43 depletion \citep{Polymenidou2011}, suggesting an important role for TDP-43 in their stability.  


%% what do I do about FUS here?
%Rare ALS-causing mutations have been found in another RNA-binding protein, FUS \citep{Vance2009-ye}, further raising the possibility that the impairment of RNA processing is a central cause of ALS.  TDP-43 and FUS have both been shown to bind a set of overlapping RNA targets \citep{Lagier-Tourenne2012-wa}.

%%TDP-43 and splicing
%%FIRST thing to be found on TDP-43. How much agreement between studies?

%"TDP-43 binds to (GU)16 repeats in apoA-II intron 2 and represses exon 3 splicing" \citep{Mercado2005}

%965 altered splicing events seen in adult mouse brain with TDP-43 depletion \citep{Polymenidou2011}
%%\citep{Tollervey2011} TDP-43 can bind within exons to repress them but can also bind up or downstream of them. Generally binding further away (up to 250nt from the exon) acts to enhance inclusion.
%
%Other splicing factors are perturbed in TDP-FTD \citep{Mohagheghi2015}: "the expression of hnRNPA1/A2 and PTB/nPTB is significantly altered in patients with frontotemporal dementiawith TDP-43-positive inclusions" A large number of splicing factors also alter the inclusion of SORT1 exon 17b.


%tDP-43 and POLDIP3 splicing - annotated exon 3 in POLDIP is enhanced by TDP, loss of TDP reduces the inclusion. Laser capture of neurons from ALS brain show a reduction in exon 3 containing POLDIP3. POLDIP3 involved in maintaining cell size. Exon3 lacking POLDIP3 cannot rescue cell size as well from POLDIP3 or TDP-43 knockdown. \citep{Shiga2012}.

%"Validation of the bona fide splicing events that are consistent data, both in neuronal and non-neuronal cell lines demonstrated that TDP-43 substantially alters the levels of isoform expression" in four genes but only in 2 these changes could also be confirmed at the protein level \citep{DeConti2015}. - Assessing splicing changes is hard and there is no guarantee that this leads to changes in protein level. 

%Repeat elements - dpn't put in introuction
%5TDP-43 binds a range of repeat elements \citep{Li2012,Zarnack2013,Kelley2014} and Alu elements in particular. Elevated levels of repeat element RNA 


\subsection{TDP-43 models of ALS/FTD }

TDP-43 continuously shuttles between the cytoplasm and the nucleus of cells \citep{Ayala2008}. However, in affected neurons and glial cells with ALS/FTD, TDP-43 leaves the nucleus and forms protein aggregates or inclusions in the cytoplasm \citep{Neumann2006-re}. Following this discovery, investigations of TDP-43 in ALS/FTD have attempted to model these changes by either depleting or overexpressing TDP-43 in the murine nervous system. There is still much debate on whether TDP-43 plays a role in neurodegeneration through a loss of nuclear function or a gain of cytoplasmic function.  A further question is the influence of rare mutations on TDP-43.
Global loss of TDP-43 is lethal in the mouse embryo \citep{Kraemer2010} and postnatal deletion leads to rapid death \citep{Chiang2010}. Conditional knockout of TDP-43 in mouse postnatal motor neurons causes a gradual degeneration of affected neurons and atrophy of muscle \citep{Iguchi2013}. Wildtype human TDP-43 has been reported as lethal when overexpressed in mouse neurons \citep{Shan2010,Wegorzewska2009} with neurodegeneration occuring in specific neuronal populations, indicating a selective vulnerability to TDP-43. Conversely, wildtype human TDP-43 was not found to be toxic when expressed at a physiological level \citep{Arnold2013} and only the ALS-causing mutant forms were found to cause motor neuron degeneration. This occurred without observed TDP-43 loss in the nucleus nor cytoplasmic aggregation, which has since been replicated \citep{Igaz2011}. Clearly neurons are very sensitive to the expression level of TDP-43 protein, and this is compounded by autoregulation. TDP-43 protein binds the UTR of TARDBP mRNA to modulate its own translation\citep{Ayala2011,Koyama2016}. Therefore, any changes in TDP-43 protein levels through knockdown or over-expression will interfere with this feedback loop, making it hard to gauge the true expression change of a particular targeting strategy. In \autoref{chapter:tdp_mice} we develop knockin TDP-43 mutant mice to overcome this hurdle and explore TDP-43 regulated splicing when the protein is expressed at physiological levels.

%%HERE is a bunch of different findings from TDP-43 animal models
%How does each support loss- or gain-of-function hypothesis?

%"knockout of transactive response DNA-binding protein 43 in mouse postnatal motor neurons using Cre/loxp system, progressive weight loss and motor impairment around the age of 60 weeks, and exhibited degeneration of large motor axon, grouped atrophy of the skeletal muscle, and denervation in the neuromuscular junction. The spinal motor neurons lacking transactive response DNA-binding protein 43 were not affected for 1 year, but exhibited atrophy at the age of 100 weeks; whereas, extraocular motor neurons, that are essentially resistant in amyotrophic lateral sclerosis, remained preserved even at the age of 100 weeks" \citep{Iguchi2013}

%"increased excitatory synaptic inputs and dendritic spine densities in early presymptomatic mice carrying a TDP-43Q331K mutation" \citep{Fogarty2016}

%"postnatal deletion of Tardbp in mice caused dramatic loss of body fat followed by rapid death. Moreover, conditional Tardbp-KO ES cells failed to proliferate" \citep{Chiang2010}

%"mice expressing a mutant frm of human TDP-43 develop a progressive and fatal neurode- generative disease reminiscent of both ALS and FTLD-U. Despite universal transgene expression throughout the nervous system, pathologic aggregates of ubiquitinated proteins accumulate only in specific neuronal populatioons, including layer 5 pyramidal neu- rons in frontal cortex, as well as spinal motor neurons, recapitu- lating the phenomenon of selective vulnerability seen in patients with FTLD-U and ALS. Surprisingly, cytoplasmic TDP-43 aggregates are not present, and hence are not required for TDP-43-induced neurodegeneration." \citep{Wegorzewska2009} Overexpression of mutant human TDP A315T in mice at levels "3-fold higher" 

%"Mice expressing hu- man TDP-43 in neurons exhibited growth retardation and prema- ture death that are characterized by abnormal intranuclear inclusions composed of TDP-43 and fused in sarcoma/translocated in liposarcoma (FUS/TLS), and massive accumulation of mitochon- dria in TDP-43-negative cytoplasmic inclusions in motor neurons, lack of mitochondria in motor axon terminals, and immature neuromuscular junctions" \citep{Shan2010} 

%\citep{Igaz2011} "Expression of either hTDP-43-ΔNLS or hTDP-43-WT led to neuron loss in selectively vulnerable forebrain regions, corticospinal tract degeneration, and motor spasticity recapit- ulating key aspects of FTLD and primary lateral sclerosis. Only rare cytoplasmic phosphorylated and ubiqui- tinated TDP-43 inclusions were seen in hTDP-43-ΔNLS mice"

%"Two ALS-causing mutants (TDP-43Q331K and TDP-43M337V), but not wild-type human TDP-43, are shown here to provoke age-dependent, mutant-de- pendent, progressive motor axon degeneration and motor neu- ron death when expressed in mice at levels and in a cell type- selective pattern similar to endogenous TDP-43. Mutant TDP-43-dependent degeneration of lower motor neurons occurs without: (i) loss of TDP-43 from the corresponding nuclei, (ii) accumulation of TDP-43 aggregates, and (iii) accumulation of insoluble TDP-4" \citep{Arnold2013}




%tTDP-43 regulates the levels of other neurodegenerative disease mRNAs like FUS and PGRN \citep{Polymenidou2011} and preferentially stabilises the expression of long intron genes, something shared with FUS. iCLIP of TDP-43 showed "unusually long clusters of TDP-43 binding at deep intronic positions downstream of silenced exons." \citep{Tollervey2011}.

%tTDP-43 and UG motifs: "TDP-43 has a unique capacity to recognize dispersed clusters of UG-rich motifs, or to spread its RNA binding to positions proximal to the UG-rich motifs" \citep{Tollervey2011} Both RRMs come together to bind UG-rich RNA \citep{Lukavsky2013} - interactions between the two RRMs are crucial for RNA binding




%%OTHER stuff TDP does


%%RNA transport
%\citep{Alami2013} finds "TDP-43 is a component of mRNP transport granules in neurons, including human stem cell-derived motor neurons, and identify a new role for TDP-43 in the cytoplasm supporting anterograde axonal transport of target mRNAs from the soma to distal axonal compartments"

%%TDP-43 autoregulation 
%first seen by Polymenidou \citep{Polymenidou2011} then assessed by \citep{Ayala2011} \citep{Er??ndiraAvenda??o-V??zquez2012} and \citep{Koyama2016}. TDP-43 binds the 3'UTR of the TARDBP mRNA to prevent the usage of the proximal polyA site. The longer TARDBP 3'UTR transcript undergoes multiple splicing rounds that ultimately reduce the levels of TDP-43 protein. With TDP-43 depletion the proximal polyA site is used and the levels of cytoplasmic TARDBP mRNA are increased.

%%TDP-43 proteomics
%proteomic study on loss of TDP-43 \citep{Stalekar2015} "TDP-43 is an important regulator of RNA metabolism and intracellular transport."
%%TDP interactome \citep{Freibaum2010-hw} shows TDP-43 interacts with multiple splicing factors (MATR3/hnRNPAB/SFPQ) as well as ribosomal proteins. "Disease-causing mutations in TDP- 43 (A315T and M337V) do not alter its interaction profile. TDP-43 interacting proteins largely cluster into two distinct interaction networks, a nuclear/splicing cluster and a cytoplasmic/translation cluster, strongly suggesting that TDP-43 has multiple roles in RNA metabolism and functions in both the nucleus and the cytoplasm"

%Impaired nuclear import of WT TDP-43 as general mechanism
%Screen of 82 nuclear import proteins "knockdowns of karyopherin-b1 and cellular apoptosis susceptibility protein resulted in marked cyto- plasmic accumulation of TDP-43." \citep{Nishimura2010} "considerable reduction in expression of cellular apoptosis susceptibility protein in frontotemporal lobar degeneration."




%Prion-like C terminal 
%
%"TDP-43 does have inter-domain in- teractions which is coordinated by the intrinsically-disordered prion-like domain" \citep{Wei2016}
%
%Mitochondria
%%TDP-43 localises to the mitochondria of motor neurons and this increases with the increase in cytoplasmic TDP in ALS. TDP-43 mutations increase mitochondrial localisation. In mitochondria TDP binds complex I components TDP-43 protein sequence contains multiple mitochondrial import sequences. Competitive inhibition of TDP-43's mitochondrial localisation rescues neurotoxicity. \citep{Wang2016}
%
%
%%TDP depletion in Alzheimers - "TDP-43 proteinopathy, initially associated with ALS and FTD, is also found in 30–60\% of Alzheimer’s disease (AD) cases and correlates with worsened cognition and neurodegeneration." Some forebrain AD sensitive neurons are selectively vulnerable to TDP-43 loss.
%




% FUS


\subsection{FUS in ALS and FTD}
Fused in sarcoma is an RNA-binding protein encoded by the \textit{FUS} gene. Over 40 mutations in FUS have been found to cause ALS, accounting for around 5\% of familial cases and 1\% of sporadic cases \citep{Vance2009-ye,Tan07102016}. FUS-ALS is distinguished from sporadic ALS by its aggressively early onset and the presence of FUS protein instead of TDP-43 in cytoplasmic inclusions. These FUS-positive inclusions can be also be seen in around 10\% of FTD cases \citep{Neumann2009} in the absence of any causative mutation. FUS is a member of the FET family of RNA binding proteins, sharing high sequence homology with EWSR1 and TAF15 \citep{Kovar2011} which have both been linked to ALS in a small number of cases \citep{Neumann2011, Couthouis2011,Ticozzi2011-bs,Couthouis2012}.

FUS as an RNA-binding protein is structurally related to TDP-43 but the overlap between their RNA targets is small \citep{Lagier-Tourenne2012-wa,Rogelj2012,Colombrita2012, Honda2014}. However it shares a lot of protein functionality with TDP-43, containing two RNA recognition motifs and a low complexity domain. It is a splicing factor, binding to GGU motifs within introns and 3' UTR sequences \citep{Rogelj2012,Lagier-Tourenne2012-wa} to enhance or repress exon inclusion and promote polyadenylation. FUS localises to the nucleus and has a number of roles in transcription, through interacting with RNA polymerase II and the spliceosome via the U1 snRNP \citep{Sun2015a, Yu2015}. Its control of polyadenylation has been shown to be through its interaction with RNA polymerase II as it can stall transcription at short 3'UTRs to encourage premature polyadenylation \citep{Masuda2015}. Like TDP-43, It can autoregulate its own translation by binding its own mRNA \citep{Zhou2013}, although the exact mechanism of this is still unclear.

% review of FUS functions citep{Ling2013}

%%FUS proteinopathies - citation?


%%TAF15 and EWSR1 share prion like domains with TDP-43 and FUS, missense variants in TAF15 in ALS patients \citep{Couthouis2011}
%%EWSR1 and TAF15 also in FTD-FUS cytoplasmic inclusions \citep{Neumann2011}
%variants in EWSR1 in ALS \citep{Couthouis2012}
%rare variants found in TAF15 in fALS causes \citep{Ticozzi2011-bs}


%Recent studies have demonstrated an interaction with RNA-polymerase II and the U1 snRNP \citep{Yu2015,Sun2015a}. 
%%SMN:
%"Expression of different FUS mutants (R521C, R521H, P525L) in neurons caused axonal defects. A protein interaction screen performed to explain these phenotypes identified numerous FUS interactors including the spinal muscular atrophy (SMA) causing protein survival motor neuron (SMN). Biochemical experiments showed that FUS and SMN interact directly and endogenously, and that this interaction can be regulated by FUSmutations" \citep{Groen2013}.
%"This studyshows that neuronal aggre- gatesformedbymutantFUSproteinmayaberrantlysequesterSMNandconcomitantlycauseareductionofSMN levels in the axon, leading toaxonal defects"
%
%It has also been linked to the minor splicesome \citep{Reber2016}. 
%
%Other roles for FUS:
%binding RNA polymerase II \citep{Schwartz2012}
%chromatin binding through binding to SAF3B and MATR3 \citep{Yamaguchi2016}
%regulates the expression of other RNA-binding proteins \citep{Nakaya2013}
%\citep{Rogelj2012}: binds GGU motifs within introns - less strongly than TDP-43 binds UG, represses inclusion of cassette exons, regulates splicing of Ewsr1 - same protein family as FUS. Genes with FUS-regulated splicing are enriched for neuronal development
%alternate polyadenylation \citep{Masuda2015} - position specific -  stalls RNA Polymerase II to initiate alternate polyadenylation of shorter transcripts
%
%Stress granules:
%	mutant FUS, but not wildtype FUS, forms stress granules in response to stress in Zebrafish \citep{Bosco2010}
%	"RNA-binding-deficient FUS strongly localized to the nucleus of Drosophila motor neurons and mammalian neuronal cells, whereas FUS carrying ALS-linked mutations was distributed to the nucleus and cytoplasm. Importantly, we determined that incorporation of mutant FUS into the SG com- partment is dependent on the RNA-binding ability of FUS" \citep{Daigle2013}
%
%\citep{Yasuda2013}: FUS recruited to APC-dependent granules which are translationally active, unlike stress granules. FUS promotes translation within these granules 
%
%Autoregulation:
%	\citep{Zhou2013} FUS binds its own introns to regulate its own expression - mutant FUS leaves the nucleus so cannot reduce its own expression levels - feedback. 

\subsection{ FUS mouse models of disease } 

In the mouse, complete knockout of endogenous FUS is lethal \citep{Hicks2000}, whereas overexpression of human FUS causes a progressive motor neuron loss and death by 3 months, accompanied by cytoplasmic FUS protein expression \citep{Mitchell2013}. This only occurred when the FUS transgene was homozygous, suggesting a dose sensitivity to the wildtype human protein for neurodegeneration. Expression of mutant human FUS in the mouse brain caused an increased cytoplasmic accumulation of FUS relative to that of the wildtype, though no neurodegeneration was observed after 3months \citep{Verbeeck2012}. 
The normal nuclear localisation of FUS can be perturbed by overexpression or by mutations but it is still unclear whether the accompanying neurodegeneration is due to a loss or gain of function. A direct comparison between FUS knockout and mutation was performed, demonstrating lethality in both conditions, but with motor neuron loss only seen in the mutant mice \citep{Scekic-zahirovic2016}. This suggests a gain of function mediated by mutant FUS being responsible for neurodegeneration. This hypothesis was bolstered by a study where human mutant FUS expression caused neurodegeneration whereas a postnatal knockout of endogenous FUS only in motor neurons did not \citep{Sharma2016}. 




\section{Aims of my PhD} % what I plan to contribute to the field - final section

The aim of my PhD is to capitalise on the RNA-seq revolution and bring together many disparate models of ALS and FTD to fully assess the nature of RNA dysregulation in disease.
I have performed a meta-analysis of published TDP-43 depletion datasets to assess the evidence of cryptic splicing, a recently discovered RNA phenotype. I am now in the middle of analysing a series of new TDP-43 and FUS associated disease mouse models generated by the Institute of Neurology. Finally, I will then use the insight gleaned from the cellular and animal models to generate novel hypotheses to explain the mechanism of ALS/FTD using RNA-seq data from human patients.

%
%
%can I use RNA-seq to gain new insight into the real human disease from the model data?
%use existing tools and develop new tools to better understand RNA processing in ALS/FTD
%
%	